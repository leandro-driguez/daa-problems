\documentclass[10pt]{amsart}

% Package Being Used:

\usepackage[utf8]{inputenc}
\usepackage{amsmath}
\usepackage{amssymb}
\usepackage{bm}
\usepackage{graphicx}
\usepackage{psfrag}
\usepackage{color}
\usepackage{hyperref}
\hypersetup{colorlinks=true, linkcolor=blue, citecolor=magenta, urlcolor=wine}
\usepackage{url}
\usepackage{algpseudocode}
\usepackage{fancyhdr}
\usepackage{mathtools}
\usepackage{tikz-cd}
\usepackage{xy}
\input xy
\xyoption{all}
\usepackage{stmaryrd}
\usepackage{calrsfs}

% Paper Format and Geometry:

\voffset=-1.4mm
\oddsidemargin=14pt
\evensidemargin=14pt
\topmargin=26pt
\headheight=9pt     
\textheight=576pt
\textwidth=441pt %441
\parskip=0pt plus 4pt

% Head Labels:

\pagestyle{fancy}
\fancyhf{}
\renewcommand{\headrulewidth}{0pt}
\renewcommand{\footrulewidth}{0pt}
\fancyhead[CE]{\fontsize{9}{11}\selectfont L. RODR\'IGUEZ Y  M. M. TIRADOR }
\fancyhead[CO]{\fontsize{9}{11}\selectfont EL BAR}
\fancyhead[LE,RO]{\thepage}

\setlength{\headheight}{9pt}

% Theorems-like Format and Numbering:

\newtheorem*{maintheorem*}{Main Theorem}
\newtheorem{theorem}{Theorem}[section]
\newtheorem{prop}[theorem]{Proposici\'on}
\newtheorem{conj}[theorem]{Conjecture}
\newtheorem{lem}[theorem]{Lema}
\newtheorem{cor}[theorem]{Corollary}
\newtheorem{teo}[theorem]{Teorema}

\theoremstyle{definition}
\newtheorem{defn}[theorem]{Definition}
\newtheorem{rem}[theorem]{Remark}
\newtheorem{example}[theorem]{Example}
\newtheorem{sol}[theorem]{Soluci\'on}
\newtheorem{prob}[theorem]{Problema}
%\newtheorem{prob}[theorem]{Problem}
\newtheorem{question}[theorem]{Question}
\numberwithin{equation}{section}
\newcommand{\lqqd}{{\small $\blacksquare$}}
\newcommand{\Proof}[2]{{\vspace{1em} $\emph{Demostración:}$ \textbf{#1} #2 \lqqd \vspace{1em}}}

% Personalized Commands:
\newcommand{\cc}{\mathbb{C}}
\newcommand{\ff}{\mathbb{F}}
\newcommand{\nn}{\mathbb{N}}
\newcommand{\pp}{\mathbb{P}}
\newcommand{\ppp}{\mathcal{P}}
\newcommand{\qq}{\mathbb{Q}}
\newcommand{\rr}{\mathbb{R}}
\newcommand{\zz}{\mathbb{Z}}

\newcommand{\ch}{\text{char}}
\newcommand{\lcm}{\text{lcm}}
\providecommand\ldb{\llbracket}
\providecommand\rdb{\rrbracket}
\newcommand\pval{\mathsf{v}_p}
\newcommand{\gp}{\text{gp}}
\newcommand{\qf}{\text{qf}}
\newcommand{\supp}{\textsf{supp}}
\newcommand{\ii}{\mathcal{Irr}}
\newcommand{\uu}{\mathcal{U}}
\newcommand{\ppf}{\mathcal{P}_{\text{fin}}}
\newcommand{\ppx}{\mathcal{P}_{\text{fin}, 0}} 
%\newcommand{\ppx}{\mathcal{P}_{\emph{fin}, \times}} % Do not use this notation
\newcommand{\inter}{\mathsf{int}}
\newcommand{\cone}{\mathsf{cone}}
\newcommand{\norm}[1]{\left\lVert#1\right\rVert}

\newcommand{\edge}[1]{\langle #1\rangle}
\newcommand{\at}{\mathcal{A}}
\newcommand{\lf}{\lfoloor}
\newcommand{\fl}[1]{\lfloor#1\rfloor}
\newcommand{\cl}[1]{\lceil #1\rceil}
\newcommand{\ds}{\displaystyle}

% \keywords{Minkowski sum, power monoid, Puiseux monoid, atomicity, factorization theory}

%\subjclass[2010]{Primary: 13C05, 13A15, 13C13; Secondary: 13A15}

\begin{document}
	
	\mbox{}
	\title{El viaje}
	
	\author{Marcos M. Tirador}
	\address{Facultad de Matem\'atica y Computaci\'on \\ Universidad de La Habana \\ Ciudad de La Habana \\ Cuba}
	\email{marcosmath44@gmail.com}
	
	
	\author{Leandro Rodr\'iguez Llosa}
	\address{Facultad de Matem\'atica y Computaci\'on \\ Universidad de La Habana \\ Ciudad de La Habana \\ Cuba}
	\email{leandro\_driguez@outlook.com}
		

\date{\today}

%\begin{abstract}
%	TODO...
%\end{abstract}

\bigskip
\maketitle


\bigskip
%%%%%%%%%%%%
%%%%%%%%%%%%

\section{Preliminares}
	Se define un grafo no dirigido $G$ como un par $(V,E)$ de conjuntos tales que el segundo es una relaci\'on ($E\subseteq V \times V$) anti-reflexiva y asim\'etrica definida sobre el primero. A los elementos de $V$ se les suele llamar v\'ertices o nodos y a los elementos de $E$ aristas. Adem\'as usaremos la notaci\'on $V(G)$ y $V(E)$ para referirnos respectivamente al conjunto de v\'ertices y al conjunto de aristas  de un grafo $G$ dado; aunque en caso de que no haya ambig\"uedad usaremos simplemente $V$ y $E$. Sea $(x,y)$ un elemento de $E$, usaremos tambi\'en la notaci\'on $\langle x,y \rangle$ para referirnos al mismo.
	
	 Decimos que la tupla $(G,w)$ es un grafo $G$ es ponderado si $w\colon E \rightarrow \rr$ es una funci\'on que asigna a cada arista un valor real, conocido como peso o costo. A la funci\'on $w$ se le llama funci\'on de ponderaci\'on. 
	
	Definimos un camino sobre un grafo $G$ como una tupla de la forma $p = \langle v_0,v_1, \dots v_k \rangle$ donde $v_i \in V$ y $\langle x_{i-1},x_i \rangle \in E$ para todo $1\leq i \leq k$. En este caso decimos que el camino $p$ conecta a $v_0$ y $v_k$. Definimos la longitud de un camino como la cantidad de aristas en el mismo. Alternativamente se usar\'a la notaci\'on $x \rightsquigarrow y$ para referirnos a un camino que comienza en $x$ y termina en $y$, e $x \rightarrow y$ para referirnos a una arista de $x$ a $y$ que es parte de un camino cualquiera que contiene a $x$ inmediatamente antes de $y$. Definimos adem\'as la distancia entre dos nodos como la longitud del camino de menor longitud que conecta a dichos nodos.
	
	En un grafo ponderado definimos la longitud de un camino como la suma de los pesos de las aristas que este contiene. Dado el camino $p$ definido anteriormente, definimos la longitud de $p$ como $\sum_{i = 1}^k w(\edge{v_{i - 1}, v_i})$. An\'alogamente definimos la distancia entre dos v\'ertices como la longitud del camino de menor longitud que los conecta.
	
\section{Enunciado del Problema}
	\begin{prob}
		El viaje
	
		Kenny y Jesús quieren hacer un viaje por carretera de La Habana a Guantánamo. Objetivo: Fiesta. Obstáculo: Precio de la gasolina. Incluyendo el punto de salida (La Habana) y de destino (Guantánamo), hay un total de $n$ puntos a los cuales es posible visitar, unidos por $m$ carreteras cuyos costos de gasolina se cononcen. Los compañeros comienzan entonces a planificar su viaje.
		
		Luego de pensar por unas horas, Kenny va entusiasmado hacia Jesús y le entrega una hoja. En esta hoja se encontraban $q$ tuplas de la forma $(u, v, l)$ y le explica que a partir de ahora considerarían como útiles sólo a los caminos entre los puntos $u$ y $v$ cuyo costo de gasolina fuera menor o igual a $l$, para $u, v, l$ de alguna de las $q$ tuplas.
		
		Jesús lo miró por un momento y le dijo: Gracias. La verdad esta información no era del todo útil para su viaje. Pero para no desperdiciar las horas de trabajo de Kenny se dispuso a buscar lo que definió como carreteras útiles. Una carretera útil es aquella que pertenece a un algún camino útil. Ayude a Kenny y Jesús encontrando el número total de carreteras útiles.
	\end{prob}	

A continuaci\'on presentamos una redefinici\'on del problema en t\'erminos m\'as formales.

	\begin{prob}
		Se tiene un grafo no dirigido y ponderado, con funci\'on de ponderaci\'on $w$, de $n$ nodos y $m$ aristas. Se tiene un conjunto $Q$ conformado por $q$ tuplas de la forma $(u,v,l)$, donde $u$ y $v$ son nodos del grafo, y $l$ es un entero no negativo. Se dice que un camino entre $u$ y $v$ de longitud $l$ es \'util si la tupla $(u,v,l') \in Q$ para alg\'un $l' \ge l$. Una arista $e$ es \'util si pertenece a alg\'un camino \'util. Encuentre el n\'umero de aristas \'utiles del grafo.
	\end{prob}

\section{Soluci\'on Propuesta} \label{opt_sol}
	
	\begin{sol}\label{sol_optima_travel} Sea $d \colon V \times V \rightarrow \rr^+$ la funci\'on que a cada par de nodos del grado hace corresponder la longitud del caminos de costo m\'inimo que los conecta. La misma puede ser hallada usando el conocido algoritmo de Floyd-Warshall, o haciendo el algoritmo de Dijkstra desde cada nodo. Luego para cada arista $\edge{x,y}$ del grafo, si existe en $Q$ una tupla $(u,v,l)$ tal que
	\begin{align}\label{condition}
		d(u,x) + d(y,v) + w \left(\edge{x, y} \right) \le l o \\
		d(u,y) + d(x,v) + w \left(\edge{x, y} \right) \le l,
	\end{align}
	entonces dicha arista puede ser contada como \'util. Para determinar esto podemos para cada arista comprobar cada una de las tuplas de $Q$. En caso de que ninguna satisfaga la desigualdad \eqref{condition}, dicha arista no ser\'a \'util.
	\end{sol}

\textbf{Complejidad temporal:}

 La complejidad temporal de la soluci\'on es $O\left(mq + \min\big(\min(q,n) m\log(m), n^3\big) \right)$. El sumando $mq$ sale del hecho de que debemos comprobar para cada arista, cada una de las tuplas de $Q$ en la condici\'on \eqref{condition}. El otro sumando es el costo de calcular la funci\'on $d$. En caso de que $m \log(m) = O(n^2)$ podemos usar la variante de hacer Dijktsra desde cada nodo, de donde sale la complejidad $O(nm\log(m))$ dado que son $n$ nodos y la complejidad de Dijkstra es $O(m\log(m))$ (remplazamos el factor $n$ por $\min(n, q)$, dado que solo nos interesan los caminos de longitud m\'inima partiendo de los nodos que aparecen en algunas de las $q$ tuplas). En otro caso podemos usar Floyd-Warshall que tiene una complejidad total de $O(n^3)$.


\textbf{Complejidad espacial:}
	La complejidad espacial est\'a determinada por el costo de almacenar la funci\'on $d$, el cual es $O(n^2)$, m\'as el costo de almacenar las tuplas y el grafo en s\'i que es $O(n + m + q)$. Por tanto la complejidad final es $O(n^2 + q)$ (siempre pueden eliminarse ciertas tuplas innecesarias de modo que siempre sea $O(n^2)$).

\begin{prop}
	La Soluci\'on \ref{sol_optima_travel} es correcta.
\end{prop}
	\Proof{}{
		Comencemos por observar que si una arista $\edge{x,y}$ satisface la desigualdad \eqref{condition} para alguna tupla $(u,v,l) \in Q$, entonces dicha arista es \'util. Esto se puede ver f\'acilmente del hecho de que el camino $u\rightsquigarrow x \rightarrow y\rightsquigarrow v$ tiene longitud $d(u,x) + d(y,v) + w\big( \edge{x,y} \big)$. Por tanto a partir de la desigualdad \eqref{condition} la longitud de ese camino es menor que $l$, de donde tenemos que es \'util para la tupla $(u,v,l)$ y as\'i la arista $(x,y)$ tambi\'en lo ser\'a.
		
		Supongamos ahora que existe una arista $\edge{x, y}$ que es \'util pero que no fue encontrada por el algoritmo. Sea $(u, v, l)$ la tupla que hace \'util a dicha arista. Por tanto, existe un camino $p_1 := u \rightsquigarrow x \rightarrow y \rightsquigarrow v$ con longitud a lo sumo $l$ o un camino $p_2 := u \rightsquigarrow y \rightarrow x \rightsquigarrow v$ con longitud a lo sumo $l$. Supongamos sin p\'erdida de generalidad que existe $p_1$. Sea $w_1$ la longitud del fragmento de $p$ que va de $u$ a $x$ y $w_2$ la longitud del fragmento de $p$ que va de $y$ a $v$. Por definici\'on de $d$, se cumple que $d(u,x) \le w_1$ y $d(y, v) \le w_2$. Entonces $d(u,x) + w(\edge{x, y}) + d(y , v) \le w_1 + w(\edge{x,y}) + w_2 = |p| \le l$. Por tanto la arista $\edge{x, y}$ satisface la desigualdad \eqref{condition} para la tupla $(u,v,l)$, lo cual contradice que el algoritmo no la encontr\'o. Podemos concluir que una arista es \'util si y solo si la soluci\'on propuesta la encuentra.
		
	}

\begin{prop}
	La Soluci\'on \ref{sol_optima_travel} es \'optima si y solo si los algoritmos de Dijkstra y Floyd-Warshall son \'optimos.
\end{prop}

	\Proof{}{
		Supongamos que dadas las entradas es mejor usar Floyd-Warshall, o sea que $n^2 = O(m log(m))$, o en otras palabras que el grafo es denso.	El caso de Dijkstra ser\'a an\'alogo. Ahora supongamos que nuestra soluci\'on es \'optima. Si existe un algoritmo mejor que Floyd-Warshall para hallar el camino de costo m\'inimo entre cada par de v\'ertices (la funci\'on $d$ enunciada en la Soluci\'on \ref{opt_sol}) entonces podremos usar ese algoritmo en lugar de Floyd-Warshall en nuestra soluci\'on para obtener una mejor, lo que contradice que fuera \'optima. 
		
		Ahora asumamos que Floyd-Warshall es \'optimo. Supongamos que existe una soluci\'on a nuestro problema que tiene complejidad mejor que la dada, para el caso peor. Entonces esta soluci\'on no utiliza Floyd-Warshall, ni ning\'un otro algoritmo que calcule la funci\'on $d$ para todo par de nodos. Por tanto habr\'an pares de nodos para los cuales $d$ ser\'a desconocida. Construyamos una entrada del problema para el cu\'al este algoritmo nuevo nunca dar\'a la soluci\'on correcta lo cual concluir\'a nuestra demostraci\'on. Asumamos que en el problema a resolver existe en $Q$ una tupla para cada par de v\'ertices del grafo. Por tanto, existe al menos una tupla para la cual la distancia entre los v\'ertices de los extremos se desconoce.  
		
	}

\medskip
%%%%%%%%%%%
%%%%%%%%%%%
%\section{Background}
%\label{sec:background}




%\newpage

%\bigskip
%%%%%%%%%%%%%%%%%%%%%%%
%%%%%%%%%%%%%%%%%%%%%%%
%\section{}





\bigskip
%%%%%%%%%%%%%%%
%\begin{thebibliography}{20}
	
%	\bibitem{FT18} Y. Fan and S. Tringali: \emph{Power monoids: A bridge between factorization theory and arithmetic combinatorics}, J. Algebra \textbf{512} (2018) 252--294.

	%	\bibitem{GGT19} A. Geroldinger, F. Gotti, and S. Tringali: \emph{On strongly primary monoids, with a focus on Puiseux monoids}, J. Algebra \textbf{567} (2021) 310--345. 

%	\bibitem{GH06} A. Geroldinger and F. Halter-Koch: \emph{Non-unique Factorizations: Algebraic, Combinatorial and Analytic Theory}, Pure and Applied Mathematics Vol. 278, Chapman \& Hall/CRC, Boca Raton, 2006.

	%	\bibitem{mR93} M. Roitman: \emph{Polynomial extensions of atomic domains}, J. Pure Appl. Algebra \textbf{87} (1993) 187--199.
	
%\end{thebibliography}


\end{document}
\@setaddresses